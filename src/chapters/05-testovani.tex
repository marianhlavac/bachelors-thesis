\chapter{Testování}\label{testovuxe1nuxed}

Pro účely testování bylo požádáno několik kolegů, aby zaujali pozici
testerů a v herně \emph{Virtualnirealita.cz} otestovali průběh výuky a
práci se spouštěčem.

Jelikož jde o aplikaci s lineárním postupem, nebyly k testování
sestaveny scénáře ani průběhy. Testování tak proběhlo zjednodušeně, kdy
každý tester byl požádán, aby prošel výukou a spustil VR aplikaci dle
jeho výběru. V průběhu jeho konání je pak pozorován a jsou průběžně
zapisovány odchylky od očekávaného chování uživatele. Následně byl
dotázán na malý počet kontrolních otázek.

\section{Předměty testování}\label{pux159edmux11bty-testovuxe1nuxed}

Z funkčních požadavků definovaných v analytické části bylo určeno:

\begin{quote}
TODO
\end{quote}

\section{Zjištěné nedostatky}\label{zjiux161tux11bnuxe9-nedostatky}

\begin{quote}
TODO
\end{quote}
