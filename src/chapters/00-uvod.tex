\section{Úvod}\label{uxfavod}

Virtuální realita (často zkracována na VR) je bezesporu novým trendem v
oblasti informačních technologií. Protože je tato technologie běžným
lidem méně dostupná, vznikly ve větších městech nové podniky, které
zprostředkovávají zážitky ve virtuální realitě za zlomek ceny celého
systému bez nutnosti znalosti systémů virtuální reality a jejich
specifikací, zajištění dostatečného výpočetního výkonu pro takové
systémy, nutnosti výběru a nákupu her pro virtuální realitu kompatibilní
s konkrétním systémem a konfigurace virtuální reality.

Takovým podnikům a jejich návštěvníkům však vznikají určité požadavky,
na které systémy virtuální reality nejsou v současné době příliš
připraveny. Uživatelské rozhraní softwaru je spíše soustředěno na
jednoho dlouhodobého uživatele, který měl prostor systému porozumět, což
je nevhodné v prostředí, kde se uživatel s virtuální realitou setkává
poprvé a v omezeném čase, po který mu byl systém zapůjčen.

\subsection{Co je virtuální
realita}\label{co-je-virtuuxe1lnuxed-realita}

Pojem virtuální realita označuje technologii prezentace prostředí pomocí
replikace lidských smyslů pro simulaci přítomnosti uživatele v takovém
prostředí. Často se virtuální realitou označuje i samotné virtuální
prostředí. Technologie tak vytvářejí iluzi reálného alternativního
světa.

Konkértní častou definici virtuální reality v angličtině: a realistic
and immersive simulation of a three-dimensional 360-degree environment,
created using interactive software and hardware, and experienced or
controlled by movement of the body můžeme volně přeložit jako
realitiská a pohlcující simulace trojrozměrného 360 stupňového
prostředí tvořeného pomocí interaktivního softwaru a hardwaru ovládaného
pohybem lidského těla.

Uživatel virtuální reality se může v prostředí typicky rozhlížet,
procházet se (v různě omezené míře) a interagovat s vyzobrazenými
objekty. Virtuální realita nalézá uplatnění v průmyslu, lékařství,
sportu, armádě a pro koncové uživatele především v zábavním průmyslu.

Počátky virtuální reality sahají až do 50. let 18. století, kdy se
experimentovalo s různými stereoskopickými displeji a klamání lidských
smyslů. Jako nejranější známý příklad virtuální reality je přístroj
zvaný \emph{Sensorama}, který byl schopen zobrazovat trojrozměrné
stereoskopické obrázky, přehrávat zvuk prostředí a vypouštět aromatické
látky, pro pohlcující zážitek.

Na počátku 20. století se objevily další příklady pohlcujících zážitků
virtuální reality. Za zmínku stojí projekt projekční místnosti \emph{The
Cave}, mezi koncové uživatele nikdy nerozšířený \emph{Sega VR Headset},
či \emph{Virtual Boy} od společnosti \emph{Nintendo}.

\includegraphics{https://upload.wikimedia.org/wikipedia/commons/c/ce/Virtual-Boy-wController.jpg}\\
\emph{fig. 1 Virtuální realita z roku 1995 -- Virtual Boy}

\subsection{Virtuální realita v
současnosti}\label{virtuuxe1lnuxed-realita-v-souux10dasnosti}

V současnosti je virtuální realita tvořena typicky pomocí počítačem
generované trojrozměrné grafiky a zvuku, snímání pohybu a snímání polohy
lidského těla. Uživateli je zážitek zprostředkován pomocí náhlavních
souprav, které vykreslují obraz, přenášejí zvuk a snímají polohu hlavy
uživatele.

V rámci této technologie tak vznikají celé systémy virtuální reality,
které disponují různými vlastnotmi, technologiemi simulace a kvalitou
simulace. Některé systémy míru a kvalitu simulace doplňují snímáním
celého lidského těla, či částí jejich končetin, např. ovladačů pro ruce.
Snímány jsou i polohy fyzických předmětů, či různých jiných ovladačů.
Systémy disponují různou technologií snímání. Některé používají
infračervené světlo a kamery, jiné systémy zase laserové snímání.

Velkou roli v této technologii hrají kvalitní a rychlé gyroskopy a
počítačový výkon. Právě kvůli virtuální realitě v poslední době začal
převažovat trend nových VR ready grafických karet, které jsou
přizpůsobeny k výpočtu obrazu z dvou úhlů pro stereoskopii.

\includegraphics{https://static3.wareable.com/media/imager/14526-b104d0dee746b81605d5ab3bc0b9c2de.jpg}\\
\emph{fig. 2 Systémy virtuální reality současnosti}

\begin{quote}
TODO: Tohle \^{} definitivně přijde nahradit vlastní fotkou, chci tam
vive, oculus a psvr pohromadě. Navíc tenhle nelze použít, nejsou práva
asi.
\end{quote}

\subsubsection{Systém HTC Vive}\label{systuxe9m-htc-vive}

Systém \emph{Vive} vyvinutý společností \emph{HTC} je jedním z
nejoblíbenějších systémů virtuální reality v současnosti. Současně s
\emph{HTC} se na vývoji podílela společnost \emph{Valve}. Tato
společnost stojí za jednou z největších platforem pro digitální
distribuci počítačových her -- službou \emph{Steam}. S touto službou je
úzce spjatá technologie \emph{SteamVR}, založená na open-source knihovně
\emph{OpenVR}. O těchto technologiích více pojednávají kapitoly níže.

Systé \emph{HTC Vive} se Skládá z náhlavní soupravy s OLED displejem o
rozlišení 2160x1200 a dvou ovladačů do ruky s gyroskopem, pěti tlačítky
a haptickou odezvou. Díky laserovému snímání je možné velmi přesně
snímat velký prostor v místnosti, ve kterém se může uživatel volně
pohybovat. V České republice je k aktuálnímu datu systém dostupný za
přibližně 24 tisíc korun. Koncovým zákazníkům se stal dostupným v dubnu
roku 2016. Pro tento systém je aplikace této práce navržena.

\includegraphics{https://upload.wikimedia.org/wikipedia/commons/7/7a/Vive_pre.jpeg}\\
\emph{fig.3 Systém virtuální reality HTC Vive a jeho ovladače}

\subsection{Platforma Steam a SteamVR}\label{platforma-steam-a-steamvr}

Protože se na vývoji systému \emph{HTC Vive} podílela společnost
\emph{Valve}, která je známá především svou platformou \emph{Steam},
určenou pro digitální distribuci her, je více než logické, že
\emph{Vive} je úzce spjat s touto platformou.

Hry a aplikace určené pro tento systém jsou primárně distribuovány skrze
platformu \emph{Steam}. Aplikace \emph{Steam} pak obsahuje
specializovanou platformu \emph{SteamVR} určenou pro práci se systémy
virtuální reality a jejich komunikaci s počítačem. V současné době
podporuje jen systém \emph{HTC Vive} a v nedávné době byla přidána
experimentální podpora systému \emph{Oculus Rift}.

Platforma \emph{SteamVR} má na starosti spojení všech ovladačů a jejich
rozpoznání a umožňuje systém z počítače ovládat (provést restart či
nastavení). V prostředí s nasazeným systémem na hlavě pak poskytuje
rozhraní pro ovládání systému, zajišťuje, abyste neopustili vyhrazený
prostor pro hraní (tzv. \emph{Chaperone bounds}) a další podpůrné funkce
důležité pro zážitky ve virtuální realitě tohoto systému.

\subsection{OpenVR}\label{openvr}

OpenVR je API rozhraní vyvíjené společností Steam, které umožňuje
snadný, multi-platformní a rychlý přístup k hardware systémů virtuální
reality různých výrobců. Poskytuje určitou míru abstrakce k tomu, aby
vývojáři měli přístup k jednotnému rozhraní bez závilosti na tom, jaký
konkrétní systém jakého výrobce právě používají.
