\section{Analýza}\label{analuxfdza}

Jak je zvykem u organizovaného vývoje softwaru -- implementací a
prototypování předchází analýza, pro stanovení požadavků uživatelů
softwaru a upřesnění funkcionalit aplikace.

Následující kapitola se takovou analýzou bude zabývat. Analyzuji
existující řešení, co požadují hráči navštěvující hernu, co požadují
zaměstnanci pracující v herně a jaké požadavky jsou realizovatelné.

\subsection{Analýza a porovnání existujících řešení
výuky}\label{analuxfdza-a-porovnuxe1nuxed-existujuxedcuxedch-ux159eux161enuxed-vuxfduky}

Výukové aplikace pro seznámení s virtuální realitou již existují.
Nicméně většina z nich trpí špatnou přístupností. Systémy jsou
navrhovány tak, aby po absolvování takové výuky již nebyly výukové
aplikace jednoduše dostupné. V našem případě jsou tak obtížně
spustitelné pro návštěvníky herny.

Takové aplikace jsou spíše určené pro toho, kdo jako první systém
konfiguruje a je jímž prvním uživatelem. Nově příchozímu k
nakonfigurovanému systému není tutoriál nabídnut a jsou přímo uvedeni do
prostředí, ve kterém se již očekává, že uživatel systém důvěrně zná.

Součásti analýzy je i porovnání jednotlivých existujících řešení, pro
které jsem si stanovil následující metriky k porovnání:

\begin{itemize}
\tightlist
\item
  jednoduchost přístupu k výukové aplikaci
\item
  rychlost a svižnost výuky
\item
  zábavnost
\item
  výstižnost
\item
  srozumitelnost
\end{itemize}

\subsubsection{SteamVR Tutorial}\label{steamvr-tutorial}

Pro totožnou platformu, pro kterou je aplikace této závěrečné práce
určena -- \textbf{HTC Vive}, existuje oficiální výuková aplikace
vytvořená přímo autory samotné platformy -- společnosti Valve.

\paragraph{Průběh výuky}\label{prux16fbux11bh-vuxfduky}

Aplikace se nejprve uvede vizuálně poutavým úvodem, který spočívá v
sestavení výukové scény animací obklopující hráče. Hráči je tak názorně
ukázana možnost rozhlížet se kolem sebe a prozkoumávat prostředí.

Následně je uživateli představena postava (\emph{Virtual Reality
Assistance and Education Core}) ze hry Portal 2, která s hráčem
komunikuje a provádí ho výukou -- stává se tak \emph{průvodcem}. Monolog
je dabovaný a v místech, kde se nachází průvodce lze číst titulky, které
jsou lokalizovány do nepřeberného množství jazyků (je k dispozici i
Čeština).

Příjemným bonusem je pro hráče hry Portal 2 familiarita postavy, která
může zvýšit hráčovu pozornost a pro hráče, kteří si hru Portal 2 v
minulosti oblíbili tak tutoriál navozuje na obličejích úsměv.

\includegraphics{http://1u88jj3r4db2x4txp44yqfj1.wpengine.netdna-cdn.com/wp-content/uploads/2016/03/Walkthrough-of-SteamVRs-new-tutorial_17-930x523.jpg}\\
\emph{fig.1 Výuková aplikace SteamVR Tutorial}

Jako první je uživateli představena plocha, tzv. \emph{Play area}, ve
které se VR zážitky budou odehrávat. Neprodleně jsou pak představeny
tzv. \emph{Chaperone bounds}, které upozorňují hráče na skutečnost, že
opouštějí hranice \emph{play area} a jsou vystaveni limitacím fyzické
místnosti, ve které se nacházejí. Protože je tato funkcionalita důležitá
v zájmu uživatele, pro jeho bezpečnost, je na ni ve výuce kladen důraz a
je proto požádán, aby se ke kraji místnosti pomalu přiblížil a následně
to stejné zopakoval na druhé straně místnosti.

Dále je uživatel požádán, aby se podíval na ovladače, které drží v ruce
a provedl s nimi libovolné pohyby pro vyzkoušení manipulace s nimi.
Poté, co se seznámí s pohyby s ovladačem jsou mu postupně představena
všechna tlačítka, která se nacházejí na ovladači a je požádán, aby každé
stiskl a vyzkoušel si tak, kde se nacházejí a jakou mají zpětnou odezvu.

Výuková aplikace je pojatá spíše komicky a každé tlačítko velmi chytře
vyvolává různé destrukční, hlasité a nečekané události, které se
průvodci nelíbí. Průvodce hráče žádá aby přestal, hráč však může mít
nutkání tyto tlačítka mačkat opakovaně, aby průvodce rozčílil a dělal
nepořádek. Tím se s tlačítky seznámí o to více.

Hráč není informován o tom, která tlačítka mají jakou funkci, logicky z
toho důvodu, že každá VR aplikace má své vlastní pojetí smyslu těchto
tlačítek. Ovšem k tlačítkům \emph{Menu} a \emph{System} je uživateli
řečeno, k čemu se nejčastěji používají.

Ke stisknutí tlačítka \emph{System} je vzápětí požádáno, což vede k
otevření \emph{SteamVR Dashboard}. Lze si povšimnout, že se hra
nepozastaví při otevření \emph{Dashboardu} a probíhá tak stále
instruktáž, jak se lze ve \emph{SteamVR Dashboard} pohybovat a k čemu je
určena.

Tím je výuka u konce, uživatel je instruován k otevření
\emph{Dashboardu} a výběrem VR aplikace. Stále ovšem může v aplikaci
zůstat a dále zkoušet práci s ovladači, nebo zhlédnout závěrečnou
animaci, kdy průvodce komicky odvezou pryč ze scény.

\paragraph{Zhodnocení}\label{zhodnocenuxed}

SteamVR Tutorial je relativně dobrým příkladem výukové aplikace. Je
kvalitně navržena, s objektivně sic strohým, ale kvalitním vizuálním a
zvukovým zpracováním.

Pro účely herny je však shledán nevhodným, jelikož je návštěvníkům herny
takový tutoriál prakticky nepřístupný. Obsluha je nucena jej spustit
manuálně a také se návštěvníka herny zeptat, jestli už tutoriál
absolvoval a zdali jej chce skutečně absolvovat. Návštěvník nemá možnost
takovou výuku zopakovat, nebo alespoň získat nějaký závěrečný přehled,
pro zopakování toho, co se naučil.

\includegraphics{http://i.imgur.com/dHgOC1S.png}\\
\emph{fig.2 Přístup k aplikaci je skryt ve SteamVR nabídce, která je
přístupna jen z monitoru počítače}

Další nevýhodou je délka tutoriálu, která se běžně pohybuje kolem 6-11
minut. To představuje v prostředí, kde se běžně systém zapůjčuje na
jednu hodinu, velkou část takového času.

\subsubsection{Oculus Touch Tutorial \& Oculus First
Contact}\label{oculus-touch-tutorial-oculus-first-contact}

Pro konkurenční systém \textbf{Oculus Rift} a jeho platformu je určena
aplikace \emph{Oculus First Contact}, která je spojena s předcházející
krátkou výukou ovladačů \emph{Oculus Touch}, které jsou k systému
\emph{Oculus Rift} prodávány odděleně.

Bez těchto ovladačů tuto výuku nelze spustit.

\paragraph{Průběh výuky}\label{prux16fbux11bh-vuxfduky-1}

Uživatel je zasazen do čistého prostředí, bez předmětů, zobrazující
pouze ovladače v ruce. Před ním se pak zobrazuje přepis (titulky) hlasu
průvodkyně, která není vizuálně zpracována, lze slyšet pouze hlas.

Jako první je požádán, aby se podíval na podlahu a zpozoroval obrys
prostoru, ve kterém se může uživatel pohybovat -- \emph{the play area}.
Následně je požádán, aby se podíval na své ruce a ovladače, které v nich
právě drží a osahal si všechna tlačítka, která se mu podaří nalézt.
Následně jsou mu všechna tlačítka jedno po druhém představeny, jsou
zvýrazněny a je požádán, aby tato tlačítka stiskl.

\includegraphics{http://i.imgur.com/pLGppB7.png}\\
\emph{fig.3 Výuková aplikace Oculus Touch Tutorial}

Pokud uživatel provádí stisky tlačítek svižně, lze tuto část projít
relativně velmi rychle. Výuka nezdržuje dlouhým monologem, nebo pauzami,
jde o krátké věty a díky tomu může působit velmi svižně.

Vzápětí jsou uživateli ovladače vizuálně z rukou odstraněny a je
požádán, aby znovu vyzkoušel mačkat tlačítka a zvedat z nich prsty tak,
aby pochopil přibližnou reprezentaci přirozeného pohybu rukou a prstů.
Je požádán, aby mačkal specifické kombinace tlačítek tak, aby vytvářel
gesta rukou jako je například gesto uzavření v pěst nebo míření na
objekty ukazováčkem.

Uživateli není vysvětlen účel tlačítek, z dříve zmíněných důvodů. Pokud
však předpokládáme, že všechny VR aplikace a hry budou implementovat
systém gest ruky, které byly vysvětleny v třetí části výuky, uživatel je
schopen odvodit účel tlačítek sám, což lze považovat za nespornou
výhodu.

Překvapující je absence objasnění smyslu tlačítek \emph{Oculus} a
\emph{Menu}, které téměř ve všech VR aplikacích fungují totožně.

Tím výuka končí a je uživateli spuštěna aplikace \emph{Oculus First
Contact}, která je určena k prohloubení seznámení s virtuální realitou a
slouží jako úvodní zábavný zážitek, který je srovnatelně kvalitní a
zábavný jako jiné herní tituly pro virtuální realitu. Tím tak lze
považovat výuku jako dokončenou a aplikací \emph{Oculus First Contact}
začíná ``zábava''.

\includegraphics{http://i.imgur.com/Ly6t8mi.png}\\
\emph{fig.4 Aplikace Oculus First Contact}

\paragraph{Zhodnocení}\label{zhodnocenuxed-1}

Oculus má výukovou aplikaci zpracovanou do podstatně rychlejšího tempa,
než SteamVR. Přispívá tomu i jednoznačné rozdělení práce od zábavy.
Nejprve přichází rychlý a strohý úvod do ovládání, který trvá přibližně
4-5 minut, až následně po tomto úvodu následuje zábavný prvek ve formě
plnohodnotného VR zážitku. Vidíme tak zásadní rozdíl vůči SteamVR, který
tyto dva prvky míchá do jednoho průběhu.

\subsection{Analýza existujících řešení
spouštěčů}\label{analuxfdza-existujuxedcuxedch-ux159eux161enuxed-spouux161tux11bux10dux16f}

Po skončení výuky můžeme předpokládat, že návštěvník herny je se
systémem do určité míry seznámen a dále je nutné mu nějakým způsobem
nabídnout výběr VR zážitků. O VR aplikacích může vědět a jít do herny za
účelem takovou aplikaci vyzkoušet, nebo může hernu navštívit z
obecnějšího důvodu -- protože chce vyzkoušet virtuální realitu.

Dvě v předchozí kapitole zmíněné platformy (SteamVR a Oculus) mají
vlastní software pro spouštění VR aplikací, které blíže analyzuji.

\subsubsection{SteamVR Dashboard}\label{steamvr-dashboard}

\emph{SteamVR Dashboard} je pouze malou modifikací již existujícího
rozhraní \emph{Steam Big Picture}, který je určen k použití \emph{Steam}
platformy z pohodlí gauče s použitím herního ovladače.

Díky tomu má \emph{SteamVR Dashboard} nespornou výhodu v tom, že většina
hráčů počítačových her se s platformou \emph{Steam} již setkala a
setkala se dokonce i s rozhraním \emph{Big Picture}, které část hráčů
dokonce používá jako primární rozhraní pro práci s platformou
\emph{Steam}.

Na druhou stranu lze však považovat jako nevýhodu ten fakt, že fakticky
\emph{Steam Big Picture} nebyl původně navržen pro použití s VR. Hráč
tak pracuje s malým oknem na virtualizovaném monitoru.

\includegraphics{https://cdn0.vox-cdn.com/thumbor/Lei26JXsB0zMdW_jkshawf3t29o=/0x205:839x677/1600x900/cdn0.vox-cdn.com/uploads/chorus_image/image/49240909/ViveTheaterModeTop.0.0.jpg}
\emph{fig.5 SteamVR Dashboard}

Z tohoto rozhraní lze procházet knihovnu her, prohlížet elektronický
obchod s hrami a hry také nakupovat, prohlížet komunitní profily,
stránky her, sledovat průběh aktuální ho stahování, používat webový
prohlížeč a přistupovat k omezenému nastavení.

Pokud budeme hodnotit \emph{SteamVR Dashboard} z pohledu návštěvníka
herny, je takové rozhraní naprosto nevyhovující. Pro návštěvníka, který
nemá s platformou \emph{Steam} zkušenosti je rozhraní spíše matoucí a
může vyžadovat nějakou dobu, než se s ním seznámí. V rozhraní se také
nacházejí sociální a komunitní funkce, které pro použití v herně nemají
žádný význam a jsou tak dalším matoucím prvkem pro návštěvníka. Zákazník
herny má nadále plný přístup k obchodu a mohl by tak na účet herny
libovolně nakupovat hry, není mu v tom nikterak zabráněno.

Seznam a výběr her je pro použití v herně taktéž spíše nevhodný. Seznam
se skládá z mřížky 3x4 grafických bannerů, které o dané hře vypovídají
jen málo. Je totiž spíše na vývojářích, či v takovém banneru zobrazí
pouze logo hry, obrázek ze hry či obojí. Lze tak velmi obtížně odhadnout
o jaký žánr hry jde, zda je zábavná, zda je subjektivně návštěvníkovi
herny vizuálně přitažlivá, jaké ovládání podporuje, či zda způsobuje
závratě a kinetózu.

\subsubsection{Oculus Home}\label{oculus-home}

\emph{Oculus Home} je plnohodnotným rozhraním pro virtuální realitu od
společnosti Oculus pro svou stejnojmennou platformu.

Skrze toto rozhraní lze procházet knihovnu her a nakupovat hry v
obchodě. Uživatel může vidět i minimální komunitní funkce, jako je
seznam hráčů a notifikace (a to i systémové).

Rozhraní je vizuálně velmi přitažlivé, zobrazuje se jako výchozí prostor
při nasazení headsetu na hlavu bez spuštěné hry, což oproti platformě
SteamVR, kde se zobrazuje prázdná šedá místnost s mřížkou, může působit
příjemně.

\includegraphics{https://i.ytimg.com/vi/kWJvFR04xyM/maxresdefault.jpg}
\emph{fig.6 Oculus Home}

Jelikož je rozhraní navržené specificky pro použití s hrami pro
virtuální realitu, k hrám lze nalézt informaci o tom, jaké ovládání
podporuje a lze je řadit podle míry ``komfortu''. Nekomfortní hry jsou
pak označeny jako takové, které mohou způsobovat kinetózu a lidé, kterým
se z intenzivnějších her dělá nevolno, se takovým hrám mohou snadno
vyhnout.

Pro hernu je však \emph{Oculus Home} také nevyhovující. Přístup k
obchodu a komunitní funkce jsou opět irelevantní a nejsou určeny
návštěvníkům herny.

\subsection{Pozorování v herně}\label{pozorovuxe1nuxed-v-hernux11b}

Ke konci března roku 2017 jsem měl příležitost stát se na jeden den
obsluhou v herně \emph{Virtualnirealita.cz} v pražských Dejvicích. Tuto
příležitost jsem využil v prospěch analýzy jako předmět pozorování a
bližšího pochopení požadavků jak hráčů tak obsluhy herny.

Jako obsluha jsem měl povinnost seznámit zákazníky s pronajatým systémem
a pokud s virtuální realitou neměli dosud zkušenost, nebo se nedoslechli
o žádné konkrétní hře, či aplikaci, kterou by si přišli vyzkoušet, byl
jsem dále pověřen doporučením nějaké hry na základě jejich preferencí.

Protože jsem se ptal na přirozené otázky, abych byl schopen lépe s lidmi
pracovat a také jim doporučit správnou hru, vznikla mi tak miniaturní
analýza z malého vzorku lidí, kteří ten den do herny dorazili.

Dotazování zákazníků se tak běžně skládalo z otázek: - ``Už jste u nás
někdy byli?'' - ``Máte zkušenosti s VR?'' - ``Hrajete počítačové hry?''
- ``Jaký žánr her rádi hrajete?''

Za daný den navštívilo hernu \textbf{15 zákazníků}, z toho \textbf{10
mužů}. Většina -- \textbf{12 zákazníků} hraje počítačové hry, ale pouze
\textbf{2 z nich} v minulosti hernu navštívili, nebo měli s VR
zkušenost. Většina z nich byla mládež (v rozmezí 15-30 let), výjimku
tvořili \textbf{2 děti} (\textless{} 15) a \textbf{2 dospělí}
(\textgreater{} 30). U jednoho z návštěvníků se projevila
\emph{kinetóza}.

Pouze \textbf{4 z nich} odpověděli, že jim nepřišly ovladače systému HTC
Vive obtížné na seznámení, stejně tak tito lidé odpověděli, že by se
obešli bez pomoci obsluhy. Dva z těchto čtyř byli ti, kteří již s VR
měli zkušenost.

\includegraphics{http://i.imgur.com/lEo1Fkh.jpg}\\
\emph{fig.3 Na jeden den jsem změnil svou pracovní roli}

Občasným jevem bylo několikanásobné vystřídání zákazníků na jednom
systému za dobu zapůjčení. To je důležitá informace, protože výuková
aplikace musí s takovým jevem počítat.

Rychlost seznámení se systémem byla převážně ovlivněna zákazníkovou
zkušenosti s počítačovými hrami.

\subsection{Aplikování bodů použitelnosti podle
Nielsena}\label{aplikovuxe1nuxed-bodux16f-pouux17eitelnosti-podle-nielsena}

Před stanovením požadavků jsem se pokusil aplikovat body použitelnosti
podle \emph{Jakoba Nielsena}, které jsem v minulosti úspěšně použil pro
návrh uživatelského rozhraní pro webové aplikace.

\emph{Jakob Nielsen} je významným odborníkem v oblasti tvorby
uživatelského rozhraní, specializující se především na webová rozhraní.
Jeho klíčovým dílem je kniha \emph{Designing Web Usability} z roku 1999,
ve které popisuje veškeré své znalosti a zkušenosti z oboru.

Na základě svých zkušeností sestavil \textbf{deset bodů použitelnosti},
které se používají pro heuristickou analýzu uživatelských rozhraní.
Jedná se o tyto body:

\begin{enumerate}
\def\labelenumi{\arabic{enumi}.}
\tightlist
\item
  Viditelnost stavu systému
\item
  Propojení systému a reálného světa
\item
  Uživatelská kontrola a svoboda
\item
  Standardizace a konzistence
\item
  Prevence chyb
\item
  Rozpoznání namísto vzpomínání
\item
  Flexibilní a efektivní použití
\item
  Estetický a minimalistický
\item
  Pomoc uživatelů pochopit, poznat a vzpamatovat se z chyb
\item
  Nápověda a návody
\end{enumerate}

Nielsenovy body použitelnosti pravděpodobně nebudou plně vhodné pro
jejich použití na aplikaci virtuální reality, jelikož jsou primárně
určeny pro analýzu webových rozhraní. V této kapitole jsem vybral ty z
nich, které nejsou vhodné a upravil je tak, aby byly aplikovatelné.

Všechny tyto body jsem pak použil pro sestavení dodatečných požadavků na
aplikaci, které lze nalézt v následující kapitole.

\subsubsection{Bod č. 2 -- Propojení systému a reálného
světa}\label{bod-ux10d.-2-propojenuxed-systuxe9mu-a-reuxe1lnuxe9ho-svux11bta}

Druhý z Nielsenových bodů použitelnosti ukazuje na to, jak by mělo
uživatelské rozhraní reflektovat známé jevy a souvislosti reálného
světa. Typickým příkladem je ikona složky vyzobrazená v rozhraní. Podle
tohoto bodu je žádoucí, aby se taková ikona opravdu podobala složce a
při přetažení objektů na takovou ikonu se provedla očekávatelná akce --
přesunutí objektu do této složky.

Ve virtuální realitě je nutné tyto vztahy ještě více utvrdit. Z
přednášky na americké konferenci GCD 2016 hovoří \emph{Kimberly Voll} ze
společnosti \emph{Riot Games} o konceptu takzvaného \emph{fidelity
contract} (volně přeloženo -- kontraktu věrnosti) a popisuje jej jako
mechanismus splnění očekávání chování virtuálního světa shodným s
chováním reálného světa.

\begin{quote}
TODO: Dopsat k tomuto tématu více.
\end{quote}

\subsubsection{Bod č. 10 -- Nápověda a
návody}\label{bod-ux10d.-10-nuxe1povux11bda-a-nuxe1vody}

Vzhledem k tomu, že samotná aplikace slouží k výuce a je ji tak možné
považovat jako nápovědu, je tento bod pro daný typ aplikace
neaplikovatelný.

\begin{quote}
TODO: Tohle zvážit a případně přepsat.
\end{quote}

\subsection{Funkční požadavky návštěvníků
herny}\label{funkux10dnuxed-poux17eadavky-nuxe1vux161tux11bvnuxedkux16f-herny}

Z pozorování v herně a analýzy existujících řešení plynou následující
požadavky vztahující se k návštěvníkům herny.

\textbf{F-A01 Uživatel se chce seznámit se základními pravidly systému
virtuální reality}\\
Uživatel chce vědět, jak se používá headset systému virtuální reality,
jak se může v \emph{play area} pohybovat, kam se nesmí vydat a jak je na
to upozorněn. Funkční požadavek je klíčový z hlediska bezpečí
návštěvníka herny a ochrany majetku herny.

\textbf{F-A02 Uživatel se chce seznámit s ovladači a jejich tlačítky}\\
Uživatel chce vědět, jak vypadají ovladače, jakými tlačítky disponují a
k čemu slouží. V závislosti na této znalosti je pak uživateli usnadněno
pochopení ovládání v konkrétních VR aplikacích.

\textbf{F-A03 Uživatel se chce seznámit s funkcemi na tlačítcích pro
konkrétní hru}\\
Uživatel chce vědět, jak se ovládá konkrétní VR aplikace.

\textbf{F-A04 Uživatel si chce vybrat VR aplikaci podle žánru}\\
Uživatel si chce zvolit VR zážitek takového žánru, který mu vyhovuje. Do
herny docházejí různé věkové a zájmové skupiny. Často záleží i na
pohlaví. Ženy většinou rády hrají méně intenzivnější zážitky, vyhýbají
se hororovým hrám a střílečkám a více ocení vizuálně atraktivní
aplikace.

\textbf{F-A05 Uživatel si chce vybrat VR aplikaci podle intenzity}\\
Uživatel, u kterého se projevuje kinetóza, si chce vybrat takovou
aplikaci, aby nebyla příliš intenzivní a jeho zážitek z VR byl
pozitivní. Ač může být toto kritérium velmi subjektivní, lze aplikace
rozdělit alespoň do dvou kategorií, jako intenzivní a klidné, kde pod
klidné aplikace spadají všechny aplikace, které mají implementovány
mechanismy zabraňující kinetóze, nebo nezahrnují pohyb kamery kinetózu
způsobující.

\textbf{F-A06 Uživatel si chce vybrat VR aplikaci podle vizuálního
zpracování}\\
Uživatel si chce vybrat takovou aplikaci, která bude pro něj vizuálně
atraktivní. Spousta uživatelů upřednostňuje určité aplikace z
jednoduchého důvodu -- líbí se jim.

\textbf{F-A07 Uživatel chce výuku kdykoliv přeskočit, nebo informace
zopakovat znova} Pokud uživatel shledá výuku subjektivně příliš
jednoduchou, či zdlouhavou, měl by mít možnost její průběh minimálně
urychlit. Naopak, pokud je pro něj výuka příliš rychlá, měl by mít na
konci výuky možnost si informace zopakovat, či zopakovat celou výuku
znova.

\textbf{F-A08 Uživatel chce, aby byla výuka časově efektivní}\\
Protože má návštěvník herny omezený čas, po který je mu zapůjčen systém
virtuální reality, je pro něj důležité, aby ho výuka o tento čas
připravila v co nejmenší míře.

\subsection{Funkční požadavky obsluhy
herny}\label{funkux10dnuxed-poux17eadavky-obsluhy-herny}

Požadavky obsluhy se velkou částí kryje s požadavky zákazníka, jen z
jiného úhlu pohledu.

\textbf{F-B01 Obsluha chce zákazníka seznámit s pravidly používání
systému virtuální reality}\\
Aby uživatel používal systém správně, obsluha se potřebuje ujistit, že
zákazník ví, jak se systém používá, aby nedošlo k jeho poškození
nesprávným použitím a zákazník nebyl vystaven nebezpečí.

\textbf{F-B02 Obsluha chce zákazníka seznámi s ovladači systému}\\
Aby uživatel byl se zážitkem spokojený, obsluha potřebuje, aby zákazník
byl schopen používat ovladače systému. Taková znalost pak zákazníkovi
usnadní pochopení ovládání konkrétních aplikací a je tak logicky více
spokojený.

\textbf{F-B03 Obsluha chce zákazníkovi vybrat VR aplikaci pro něj
vhodnou}\\
Zákazníci velmi často přicházejí do herny pouze za účele vyzkoušení
virtuální reality. Zřídkakdy se stává, že by zákazník věděl o jakou
konkrétní VR aplikaci má zájem a chce si ji vyzkoušet. Obsluha je tak
povinna zjistit, co bude zákazníkovi vyhovovat a vybrat mu tak nějakou
aplikaci či herní titul pro něj vhodný.

\textbf{F-B04 Obsluha chce zákazníka upozornit na blížící se konec
vypůjčení systému}\\
Přibližně pět minut před koncem doby zápůjčky obsluha žádá zákazníka,
aby si na moment sundal sluchátka a mohla jej upozornit na blížící se
konec.

\subsection{Funkční požadavky
obecné}\label{funkux10dnuxed-poux17eadavky-obecnuxe9}

Požadavky nekategorizovatelné jako požadavek návštěvníka či obsluhy
herny. Valná většina z nich se týká fukncionality spouštěče.

\textbf{F-C01 Aplikace chce zobrazit uživateli seznam VR aplikací a
umožnit mu výběr}\\
Základní funkce spouštěče je zobrazení seznamu VR aplikací, ze kterých
může uživatel provést výběr. Takový seznam by měl poskytovat možnost
vyhledávat přímo podle názvu, dále podle žánru, intenzity i podle
vizuálu.

\textbf{F-C02 Aplikace chce stáhnout data o VR aplikaci}\\
Aby mohla aplikace splnit požadavek \emph{F-C01}, je nutné taková data o
hrách získat. Většina požadavkem zmíněných dat je dostupná přes veřejná
API. Více se získáním dat bude zabývat návrh.

\textbf{F-C03 Aplikace spustí uživatelem vybranou VR aplikaci}\\
Poté, co uživatel provede výběr aplikace, je tato aplikace spuštěna a
funkce spouštěče jsou pozastaveny či ukončeny.

\textbf{F-C04 Po ukončení VR aplikace je uživateli znovu nabídnut
přehled her}\\
Po ukončení práce s VR aplikací, kterou uživatel spustil, je mu opět
nabídnut výběr spouštěče (pokračováním v činnosti či opětovým
spuštěním).

\subsection{Nefunkční požadavky}\label{nefunkux10dnuxed-poux17eadavky}

\textbf{N-01 Aplikace je navržena pro systém HTC Vive}\\
Ze zadání plyne soustředění aplikace na jednu platformu a její konkrétní
ovladače.

\textbf{N-02 Aplikace je vizuálně atraktivní}\\
Aby byl uživatelův dojem z aplikace pozitivní a příjemný, měla by
aplikace splňovat alespoň nějakou základní úroveň kvality vizuálního
zpracování.

\textbf{N-03 Výukou je uživatel prováděn mluvenou řečí}\\
Jelikož je kvůli disperzi krajů obrazu, omezenému rozlišení a obtížně
proveditelnému umístění psaného textu ve virtuální realitě, je nutné
kromě titulků ve formě takového textu, uživatele navigovat i
prostřednictvím mluveného slova. Požadavek na primární jazyk mluveného
slvoa je Čeština.

\textbf{N-04 Výuka je časově efektivní}\\
Protože je návštěvník herny časově omezen dobou zapůjčení systému, je
nutné, aby taková výuka trvala co nejkratší možnou dobu.

\textbf{N-05 Aplikace bude jednoduchá na použití}
