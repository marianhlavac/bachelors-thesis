\section{Návrh}\label{nuxe1vrh}

\subsection{Průběh výuky}\label{prux16fbux11bh-vuxfduky}

Před konkrétním návrhem přesného scénáře výuky se v této kapitole budu
nejdříve zabývat hrubým a obecnějším návrhem průběhu celé výuky.
Samotnou výuku rozdělím do tzv. momentů, které si i označím
identifikátory, pro snazší pozdější referenci v textu. Výsledný průběh
pak dále zpracuji ve formě storyboards.

Celá výuka bude v češtině, jelikož se prozatím počítá pouze s nasazením
v české herně. Do herny však chodí i zahraniční návštěvníci a s nimi je
nutné počítat taky. Rámec této závěrečné práce nebude zahrnovat překlad
do jiného jazyka, ovšem aplikace bude na lokalizaci připravena.

\subsection{Momenty výuky}\label{momenty-vuxfduky}

Seznam klíčových momentů průběhu výuky v chronologickém pořadí:

\begin{itemize}
\tightlist
\item
  M1 Uživatel je uvítán do herny, kterou navštívil.
\item
  M2 Uživateli je vysvětleno, jaký bude průběh výuky.
\item
  M3 Uživateli je umožněno výuku přeskočit.
\item
  M4 Uživateli je představena play area a chaperone bounds.
\item
  M5 Uživatel je požádán, aby prozkoumal ovladače systému.
\item
  M6 Uživateli je představeno každé tlačítko na ovladači a je požádán
  aby je stiskl.
\item
  M7 Uživateli je vysvětleno, k čemu je určen spouštěč, který je mu
  zobrazen po skončení výuky.
\end{itemize}

\subsubsection{M1 Uvítání herny}\label{m1-uvuxedtuxe1nuxed-herny}

Jako první moment je herně posktynut velmi krátký prostor na její
prezentaci. Návštěvník herny je uvítán jménem herny do virtuální
reality. Zároveň je mu na krátký moment zobrazeno logo herny. Díky
tomuto prvku je aplikace blíže svázána s hernou a jedná se tak i o
jistou formu brandingu.

Díky tomuto prvku mohou mít o aplikaci zájem i jiné firmy provozující
herny virtuální reality, jelikož prozatím neexistuje žádná jednoduše
dostupná aplikace pro virtuální realitu, která by vhodným způsobem
prezentovala hernu, kterou uživatel právě navštívil.

\subsubsection{M2 Vysvětlení průběhu
výuky}\label{m2-vysvux11btlenuxed-prux16fbux11bhu-vuxfduky}

Aby byl uživatel připravený a věděl, co jej čeká, je mu velmi stručně
přiblížen průběh výuky. Zároveň se tak může lépe v následujícím momentu
rozhodnout, zda-li bude výuku přeskakovat, či nikoliv.

\subsubsection{M3 Možnost přeskočení
výuky}\label{m3-moux17enost-pux159eskoux10denuxed-vuxfduky}

Protože někteří uživatelé jsou již systému HTC Vive znalí, je jim
umožněno takovou výuku přeskočit a jsou rovnou uvedeni do spouštěče.

Přeskočení výuky lze provést stisknutím kombinace tlačítek na ovladači.
Na obou ovladačích bude uživatel nucen stisknout tlačítko spouště a
tlačítko nabídky. Tato kombinace bude uživateli představena a jde o
takovou kombinaci tlačítek, kterou neznalý uživatel omylem nestiskne.

Stisknutí těchto dvou tlačítek není úplně přirozené. Při běžném držení
má uživatel palec spíše v oblasti nad dotykovou plochou, nebo mírně pod
ní. Tlačítko nabídky se nachází nad dotykovou plochou. Zároveň stisknutí
obou tlačítek zároveň je mírně nepřirozené (je nepravděpodobné, že by
tato dvě tlačítka uživatel stiskl omylem, např. nevhodným úchopem
ovladače), ale zároveň ne nemožné, či obtížně stisknutelné.

Původním záměrem bylo instruovat uživatele k této kombinaci tlačítek
pouze slovy a textem -- tím by bylo zajištěno, že uživatel, který
tlačítka a jejich pojmenování nezná, nebude schopen tato tlačítka najít
a stisknout. V závěru jsem však došel k tomu, že je naprosto v
kompetenci uživatele, aby se rozhodl, jestli chce výuku spustit, či
nikoliv, a tedy pokud se rozhodne výuku nepodstoupit i za předpokladu,
že tlačítka nezná, je to jeho volba, která bude respektována. Zároveň
může dojít i k jazykovým nepřesnostem a uživatel může z jiných zdrojů
znát tlačítka pod jinými názvy, nebo jejich pojmenování znát pouze ve
specifické anglické mutaci (mnohem pravděpodobnější).

\subsubsection{M4 Představení Play area a Chaperone
Bounds}\label{m4-pux159edstavenuxed-play-area-a-chaperone-bounds}

Jelikož bezpečnost zákazníka a majetku herny je prioritní, jsou
uživateli vysvětleny pravidla pohybování se v play area jako první.

Uživateli se na podlaze zobrazí ohraničení odpovídající velikosti a
pozici nastavené play area. Je mu vysvětleno, k čemu play area slouží.
Následně jsou mu představeny chaperone bounds. Uživatel je požádán, aby
k nim přistoupil, aby si jejich funkcionalitu vyzkoušel.

\includegraphics{http://icdn9.digitaltrends.com/image/chaperone-beginner-1500x1000.png}
\emph{fig. 1 Chaperone bounds a jejich možnosti v nastavení OpenVR}

Tento přístup je tak inspirován výukou v aplikaci \emph{SteamVR
Tutorial}, nicméně je zkrácena o jedno opakování, aby tato část nebyla
příliš dlouhá. Ač jsem výše poznamenal, že je žádoucí, aby byl na tuto
část kladen důraz, domnívám se, že jedno přistoupení k mřížce uživateli
stačí k tomu, aby intuitivně funkci pochopil a na zobrazení mřížky v
budoucnu reagoval.

\subsubsection{M5 Prozkoumání
ovladače}\label{m5-prozkoumuxe1nuxed-ovladaux10de}

Uživatel je požádán, aby si prohlédl ovladač. V aplikaci bude
vykreslován velmi přesný model ovladače, takže uživatel může pohodlně
prozkoumat, jak ovladač vypadá, pokusit se nalézt tlačítka a najít
správný úchop pro vlastní komfort.

Délka této části je dynamická a uživateli je tady poskytnut libovolný
prostor pro jeho vlastní zkoumání. Důvod je prostý -- každému z
uživatelů bude průzkum ovladačů trvat různou dobu, odvíjející se i od
důkladnosti prozkoumávání. Zde by mělo být kompletně na jeho uvážení,
zda-li si chce a potřebuje ovladače prohlédnout skutečně důkladně, nebo
mu stačí jen letmý pohled na jejich vzhled a přibližnou polohu tlačítek.

Pokud se uživatel rozhodne pro druhou volbu, neznamená to, že by se
nedosáhlo na potřebnou kvalitu výuky. Seznámení s tlačítky zajistí
moment \emph{M6}, takže se nestane, že by s ovladači po skončení výuky
nebyl seznámen pouze proto, že si ovladače v této části neprohlédl
důkladně.

Uživatel je pak požádán, aby ve chvíli, kdy bude připraven pokračovat ve
výuce dále, si stoupl do středu místnosti a zvedl své ovladače před sebe
tak, aby na ně viděl.

Skriptově nebude podmíněno, aby si uživatel stoupl do středu místnosti.
Instrukce slouží spíše k úpravě pozice uživatele. Poslední jeho poloha
je totiž u kraje místnosti, kde právě před chvílí zkoumal Chaperone
bounds. Ač mají Chaperone bounds od reálně vyměřené velikosti místnosti
určitý odstup, stále se může stát, že uživatel ve chvíli, kdy je
požádán, aby zvedl natažené ruce před sebe, natáhne ruce příliš a může
ovladačemi praštit do stěny, která je před ním.

\subsubsection{M6 Představení
tlačítek}\label{m6-pux159edstavenuxed-tlaux10duxedtek}

Každé tlačítko je uživateli postupně představeno, zvýrazněno se
zobrazením titulku názvu tlačítka a je požádán aby tlačítko stiskl, nebo
s ním provedl nějakou interakci v následujícím pořadí:

\begin{itemize}
\tightlist
\item
  Tlačítko spouště
\item
  Boční tlačítka
\item
  Dotyková plocha
\item
  Tlačítko nabídky a systémové tlačítko
\end{itemize}

Po každém požádání o stisk tlačítka je uživatel jen přibližně seznámen s
tím, k čemu se tlačítko běžně používá. Je však nutné u návrhu scénáře
dát pozor, aby takové informace nebyly zavádějící, protože jak jsem již
v analýze zmínil, tlačítka si VR aplikace mapují podle vlastního uvážení
a každá aplikace tak tlačítka používá k různě jiným činnostem.

\includegraphics{http://i.imgur.com/5rTX05h.png} \emph{fig. 2 Náčrt
ovladače HTC Vive}

\begin{quote}
TODO: Nahradit tento obrázek jiným, pravděpodobně nejsou práva na
použití!
\end{quote}

Protože je tlačítko spouště důležité, je do výuky přidán prvek
laserového ukazovátka, pomocí kterého se uživatel naučí s ovladačem
mířit a spouští vybírat. Mělo by to tak podvědomě utvrdit často VR
aplikacemi využívaný koncept navigace v nabídkách -- ukázáním a výběrem
spouští.

Následně je uživatel požádán, aby tlačítko spouště stiskl, což vyvolá
akci ve formě zesílení laserového paprsku a vypalování tmavých stop do
okolí po tomto paprsku. Uživatel je požádán, aby namířil na terč, který
bude pro účely tohoto kroku do scény umístěn a stiskl spoušť. Je tak v
rychlosti uveden do schopnosti mířit ovladačem a zároveň provést akci.

Protože laserový paprsek vypaluje do okolí stopy, uživatel může být
podvědomě podnícen práci se spouští procvičit více a to tak, že začne do
okolí vypalovat další stopy. Ač je hned následně pobídnut, aby stiskl
boční tlačítko, může tak stále činit a dále používat tlačítko spouště s
laserovým paprskem.

Poté, co uživatel stikne boční tlačítko, je mu představena dotyková
plocha. Je důležité, aby uživatel pochopil, že má dvě funkce. Pohyb
prstem přes plochu a její stisknutí. Na ovladači přes dotykovou plochu
se zobrazí barevný kruh a uživatel je požádán, aby přes plochu přejížděl
prstem, vybral si barvu a dotykovou plochu stiskl. Výběrem barvy a
stikem dotykového tlačítka je uživateli změněna barva laserového
ukazovátka. Opět jej to může podpořit v další spontánní činností s
ovladači.

Jako poslední jsou mu představena tlačítko nabídky a systémové tlačítko.
Tady uživatel výjimečně není požádán o stisk těchto tlačítek, jelikož
systémové tlačítko otevírá \emph{SteamVR Dashboard}, který záměrně
uživateli představit nechceme. V rámci konzistence pak nechceme
uživatele žádat o stisk tlačítka nabídky.

Nicméně uživateli je vysvětlena funkce obou tlačítek a nedochází ve
výuce ke snaze přesvědčit jej, aby systémové tlačítko nemačkal. Místo
toho je mu vysvětleno, k čemu slouží, co se po stisku stane a velmi
mírně s dobrou volbou slov doporučíme procházení \emph{SteamVR
Dashboard} pouze zkušenějším uživatelům. Chceme však toto tlačítko
vysvětlit i uživatelům neznalým, aby při stisku tohoto tlačítka
nepanikařili a vzpomněli si na výuku, která je naučila stisk tohoto
tlačítka v případě, že nějakou systémovou nabídku otevřeli omylem.

\subsubsection{M7 Představení
spouštěče}\label{m7-pux159edstavenuxed-spouux161tux11bux10de}

Po skončení výuky je uživateli oznámeno, že je to vše, co o systému
potřebuje vědět a je mu vysvětleno, co se bude dále dít.

Krátce je mu představeno, co před sebou vidí, k čemu je spouštěč určen a
jak může spustit svůj první VR zážitek.

\subsection{Návrh výuky}\label{nuxe1vrh-vuxfduky}

Poté, co jsem specifikoval hrubý návrh scénáře výuky a její momenty, lze
z těchto momentů sestavit konkrétní podobu výuky.

\subsubsection{Návrh scénáře}\label{nuxe1vrh-scuxe9nuxe1ux159e}

První část návrhu výuky je sestavení scénáře, který pak lze velmi
efektivně využít pro skriptování průběhu, zobrazení přepisu a samotnému
dabování mluveného slova.

Ve scénáři jsou uvedeny identifikátory momentů. Označují části, které
vycházejí ze zmíněných momentů.

Tento konkrétní přepis scénáře je lokalizován do češtiny a je určen pro
použití v herně \emph{Virtualnirealita.cz}, kde bude později produkčně
nasazen. Aplikaci bude však možné připravit i pro jiné herny, nebo jiná
prostředí. Bude však nutné pozměnit tento scénář, konkrétně v místech
úvodu a závěru.

\begin{center}\rule{0.5\linewidth}{\linethickness}\end{center}

\emph{Zobrazí se logo herny. {[}M1{]}}

\textbf{Průvodce:} Vítejte v herně Virtualnirealita.cz! Tato krátká
výuka vás provede vstupem do virtuální reality. \emph{{[}M2{]}}

\emph{Zobrazí se instrukce pro přeskočení výuky na obrazovce.}

\textbf{Průvodce:} Pokud s virtuální realitou již máte zkušenosti a
výuku chcete přeskočit, stiskněte kombinaci tlačítek menu a spouště.
\emph{{[}M3{]}}

\emph{- krátká pauza -}

\textbf{Průvodce:} Rozhlédněte se kolem sebe a na podlahu. Ohraničení,
které můžete vidět na zemi je místo, ve kterém se lze volně pohybovat v
průběhu vašich zážitků ve virtuální realitě. \emph{{[}M4{]}}

\textbf{Průvodce:} Toto ohraničení však není vidět vždy, proto se
zobrazuje pomocná mřížka, která vás na tyto hranice upozorní, pokud se
je pokusíte překročit.

\textbf{Průvodce:} Nyní se zkuste pomalu k hranici přiblížit, abyste si
vyzkoušeli, jak to funguje. Za tuto hranici dále nechoďte.

\emph{- krátká pauza -}

\textbf{Průvodce:} Až dokončíte prozkoumávání ohraničení, vraťte se
doprostřed místnosti a zvedněte obě ruce před sebe. Řekneme si něco k
ovladačům, které máte v ruce. \emph{{[}M5{]}}

\emph{Čekání na reakci uživatele: zvednutí rukou před sebe\ldots{}}

\textbf{Průvodce:} Toto jsou ovladače systému HTC Vive. Nyní vám
představíme tlačítka těchto ovladačů. \emph{{[}M6{]}}

\emph{Zvýrazní se tlačítko spouště.}

\textbf{Průvodce:} Toto je tlačítko spouště. Slouží ve hrách většinou
jako tlačítko pro výstřel, nebo výběr položky v menu.

\emph{Uživateli začne z pravého ovladače vyzařovat slabý laserový
paprsek.}

\textbf{Průvodce:} Namiřte laserové ukazovátko na terč a stiskněte
spoušť.

\emph{Čekání na reakci uživatele: namíření na terč a stisk
spouště\ldots{}}

\emph{Zvýrazní se boční tlačítko.}

\textbf{Průvodce:} Perfektní! Toto je boční tlačítko, které lze
stisknout z libovolné strany. Slouží jako alternativní funkce, například
pro přebíjení zbraně. Stiskněte jej.

\emph{Čekání na reakci uživatele: stisk bočního tlačítka\ldots{}}

\emph{Zvýrazní se dotyková plocha.}

\textbf{Průvodce:} Dobře. Toto je dotyková část ovladače. Můžete po ní
přejíždět prstem a také ji stisknout. Přejížděním prstu vyberte barvu a
stisknutím dotykové plochy vyberte barvu svého laserového ukazovátka.

\emph{Čekání na reakci uživatele: stisk dotykové plochy\ldots{}}

\textbf{Průvodce:} Skvěle! Zbývají nám dvě tlačítka. Menu tlačítko a
systémové tlačítko.

\emph{Zvýrazní se menu tlačítko.}

\textbf{Průvodce:} Menu tlačítko slouží většinou k vyvolání nabídky ve
hře. Často můžete tímto tlačítkem také pozastavit hru.

\emph{Zvýrazní se systémové tlačítko.}

\textbf{Průvodce:} Systémové tlačítko pak otevírá rozhraní systému
Steam. Pokud jste s platformou Steam seznámeni, můžete toto tlačítko
používat pro procházení knihovnou. Stejným tlačítkem toto rozhraní i
můžete zavřít.

\textbf{Průvodce:} Nyní jste připraveni spustit svůj první zážitek ve
virtuální realitě. Před sebou vidíte knihovnu dostupných aplikací naší
herny. Vyberte si, o který zážitek máte zájem, namiřte na něj a
stiskněte spoušť. \emph{{[}M7{]}}

\begin{center}\rule{0.5\linewidth}{\linethickness}\end{center}

\subsubsection{Storyboards}\label{storyboards}

Pro lepší vizualizaci je k podrobnému konkrétnímu scénáři i ilustrován
průběh výuky ve formě storyboards.

\begin{quote}
TODO: Storyboards
\end{quote}

\subsection{Návrh spouštěče}\label{nuxe1vrh-spouux161tux11bux10de}

Spouštěč je funkcionalita aplikace navazující po výuce. Je určen k tomu,
aby nahradil stávající řešení výběru VR aplikací skrze \emph{SteamVR
Dashboard}, které se ukázalo být nevhodné pro použití v prostředí herny.

Podle funkčních požadavků a v kontrastu s existujícími řešení v podobě
\emph{SteamVR Dashboard} a \emph{Oculus Home} chceme vytvořit takový
spouštěč, který bude pro uživatele jednoduchý, bude brát v potaz fakt,
že uživatel může být v systému virtuální reality stále nováček a že
nemusí znát tituly podle jejich názvu. Nechceme uživatele zatěžovat v
herně nerelevantními komunitními funkcemi a nechceme uživateli jednoduše
dovolit prohlížet obchod a nakupovat tituly na účtu herny. Zároveň tato
funkce nahrazuje nutnost obsluhy dotazovat se návštěvníků, co mají rádi
a odhadovat tak o jaký typ zážitku by tak mohli mít návštěvníci zájem.

\subsubsection{Návrh rozhraní}\label{nuxe1vrh-rozhranuxed}

Pro rozhraní lze využít celý prostor kolem uživatele. Nebude se jednat o
rozhraní, které můžeme vidět u \emph{SteamVR Dashboard} -- ploché
dvourozměrné rozhraní vykreslované na malou plochu před uživatelem.

Základní myšlenka rozhraní je přímý přístup k výběru VR aplikací jako
hlavní primární obrazovka spouštěče. Oba zkoumané existující řešení
zmíněné výše mají výběr VR aplikací ukrytý pod tlačíkem ``Library''.

Jako první bude uvádět rozhraní velký nadpis vyzývající uživatele k
činnosti: ``Vyberte si VR aplikaci''. Pod ním bude zobrazen název
aktuálně otevřené kategorie s šipkou evokující známý dropdown prvek,
kterým může uživatel změnit aktuálně zobrazenou kategorii. Pod výběřem
kategorií se nachází již samotná mřížka s aplikacemi. Mřížka má na výšku
čtyři prvky a na šířku počet sloupců dynamický, podle velikosti
místnosti takový, že vyplní bannery pokud možno kruh okolo uživatele.

\includegraphics{http://i.imgur.com/EEyrMmf.png}\\
\emph{fig. 3 Rozložení prvků rozhraní kolem uživatele}

VR aplikace budou v mřížce zobrazovány velmi podobně, jako jsou
zobrazovány v existujících spouštěčích -- vizuální obdélníkový banner s
vizuálem hry. Velký rozdíl se však bude projevovat při ukázání
ukazatelem na takový banner. Hra zobrazí svůj rychlý detail. Místo
vizuálního banneru zaujme krátké video pořízené ze hry (tzv. in-game
gameplay), které se bude opakovat. Nepůjde tedy o vizuál autorů
aplikace, ani o trailer, ale čistě realistický záznam přímo ze hry.
Uživatel tak bude schopen velmi přesně odhadnout, o čem aplikace je,
jaká je její vizuální úroveň a přibližně i hratelnost a celkový dojem z
aplikace, ještě dřív, než ji spustí. Napravo od videa bude pak doplněno
celým názvem titulu, krátkým popisem a kategorizací podle žánru a
intenzity. Pokud se bude jednat o často spouštěnou aplikaci, bude
automaticky označena jako oblíbená. Celý tento blok detailu hry bude k
uživateli mírně přiblížen a ostatní prvky se stanou částečně průhledné a
budou mírně potlačeny do pozadí.

\includegraphics{http://i.imgur.com/W7i1O7H.png}\\
\emph{fig. 4 Základní stav aplikace spouštěče}

Mřížka těchto bannerů se bude zobrazovat v kruhu okolo uživatele. Hlavní
mřížka aplikací bude zarovnána k pravému ``virtuálnímu okraji'', za
kterým budou dva sloupce dalších bannerů, označených jako ``Naše herna
doporučuje''. Tyto bannery bude volit herna jako doporučené hry pro své
zákazníky a bude obsahovat maximální počet 8 aplikací. Pravý ``virtuální
okraj'' se bude nacházet po pravé ruce uživatele a hlavní mřížka
aplikace se bude podle počtu zobrazených aplikací rozšiřovat proti směru
hodinových ručiček o obvodu kruhu, na kterém se mřížka zobrazuje. Pokud
počet aplikací bude větší, než prostor k zobrazení bannerů na mřížce,
zobrazí se pod mřížkou přepínač stránek.

\includegraphics{http://i.imgur.com/K37enUq.png}\\
\emph{fig. 5 Detail hry po ukázání na jeho položku v mřížce}

Po výběru prvku pro změnu kategorie se potlačí pozadí stejným způsobem,
jako při práci s detailem hry. Do popředí se pak zobrazí velmi
jednoduchá nabídka v podobě seznamu dostupných kategorií, ze kterých
může uživatel vybírat. První oddělená položka této nabídky bude tlačítko
pro návrat nazvané ``Zpět''.

\includegraphics{http://i.imgur.com/49YOKw4.png}\\
\emph{fig. 6 Výběr kategorie po kliknutí na prvek výběru kategorie}

Rozhraní by tak mělo být velmi přehledné a především jednoduché.
Uživatel se v rozhraní nemá kde ztratit, rozhraní nemá přechod na jiné
obrazovky či stavy s vyjímkou možnosti třídění her.
