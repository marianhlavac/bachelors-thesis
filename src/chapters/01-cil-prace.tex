\chapter{Cíl práce}\label{cuxedl-pruxe1ce}

Cílem práce je vytvořit aplikaci, která usnadní návštěvníkům
herny seznámení se s virtuální realitou a díky tomu také práci obsluze herny.
Výsledná práce je určená k tomu, aby suplovala klíčové nedostatky
systému, na které lze narazit při použití v prostředí herny.

Aplikace bude rozdělena na dvě části -- na část výukovou a spouštěč.

Výuková část provede návštěvníka vstupem do virtuální reality. Představí mu prostředí, ve kterém se bude pohybovat a ukáže mu ovladače systému \emph{HTC Vive}, pro který je tato aplikace
primárně určena. Návštěvník herny tak bude v co nejkratším čase zaučen a
zefektivní se i práce obsluhy herny, která se může začít věnovat dalším zákazníkům.

Po ukončení výuky bude pomocí vlastního spouštěče představena nabídka
titulů, které si může návštěvník v herně vyzkoušet. Knihovna her a
aplikací bude kategorizována a bude zobrazovat tituly, které
herna doporučuje, či tituly, které jsou v herně oblíbené. Aplikace tím nahrazuje také povinnost obsluhy dotazovat se návštěvníků, co mají rádi a
odhadovat, o jaký typ zážitku by mohli mít zájem.

\newpage

Výstupem bude analýza existujících řešení, výsledky analýzy v podobě
požadavků na aplikaci, návrh aplikace pro výuku společně s návrhem
spouštěče a v neposlední řadě samotná realizace takové aplikace.

Mezi plánované klíčové vlastnosti aplikace patří důraz na efektivitu výuky,
kvalitní vizuální zpracování a nízká obtrusivita\footnote{obtrusivní -- vnucující se, nápadný, vtíravý} aplikace. Je nutné
myslet na to, že aplikace bude nasazena v prostředí, kde návštěvníci
disponují limitovaným časem -- zakoupili si omezený čas zápůjčky systému
a aplikace by je neměla o jejich zakoupený čas připravit.
