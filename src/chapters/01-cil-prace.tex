\section{Cíl práce}\label{cuxedl-pruxe1ce}

Cílem této závěrečné práce je vytvořit aplikaci usnadňující návštěvníkům
herny seznámení s virtuální realitou a usnadní práci obsluze herny.
Výsledná práce by tak měla nahrazovat klíčové nedostatky systému, na
které lze narazit při použití v prostředí herny.

Aplikace bude rozdělena na dvě části -- výukovou a spouštěč.

Výuková část provede uživatele vstupem do virtuální reality. Bude mu
představeno prostředí, ve kterém se bude pohybovat a budou mu
představeny ovladače systému \emph{HTC Vive}, pro který je tato aplikace
primárně určena. Návštěvník herny tak bude v co nejkratším čase zaučen a
zefektivní se i práce obsluhy herny. Zatímco návštěvník prochází výukou,
obsluha se může začít věnovat dalším zákazníkům herny.

Po ukončení výuky bude pomocí vlastního spouštěče představena nabídka
titulů, které si může uživatel v herně vyzkoušet. Knihovna her a
aplikací bude taktéž kategorizována a bude zobrazovat oblíbené tituly,
které herna dopouručuje, či ty tituly, které jsou v herně oblíbené.
Funkce tak nahrazuje dotazovat se návštěvníků, co mají rádi a odhadovat
tak o jaký typ zážitku by mohli mít zájem.

Výstupem bude analýza současného stavu, návrh aplikace pro výuku a
doplňkové aplikace spouštěče a následně samotná realizace takové
aplikace.

Mezi plánované klíčové vlastnosti aplikace je zařazena efektivita výuky,
kvalitní vizuální zpracování a nízká obtruzivnost aplikace. Je nutné
myslet na to, že aplikace bude nasazena v prostředí, kde uživatelé
disponují limitovaným časem. Zakoupili si omezený čas zápůjčky systému a
aplikace by je neměla o jejich zakoupený čas připravit.
