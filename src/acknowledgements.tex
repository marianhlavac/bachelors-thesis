Děkuji svému vedoucímu práce, panu Jiřímu Chludilovi, za jeho pomoc a vstřícný přístup a věcné rady při vedení této práce. Děkuji také herně Virtualnirealita.cz za zapůjčení vybavení a poskytnutí užitečných dat o zákaznících pro analýzu.